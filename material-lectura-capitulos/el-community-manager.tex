\chapter{El community manager}

Con la implementación de la \href{https://en.wikipedia.org/wiki/Web\_2.0}{Web 2.0} se vive una comunicación bidireccional, donde se destaca lo dinámico, lo activo, la interacción y la inmediatez. Esto ha influenciado fuertemente en la manera de hacer marketing. Las redes sociales son una herramienta perfecta para llevar a cabo estrategias efectivas y tienen un papel importante en la conversión de clientes, ya que generan mayor confianza y por ende, motivación para concretar compras.

El “community manager” es responsable de generar contenido que incite la participación de la comunidad y el crecimiento de la misma. También es el encargado de mantener la relación con los clientes, responder preguntas, dudas, resolver los problemas que los usuarios puedan tener y pasar la retroalimentación de los clientes a otras áreas de la empresa.

Entre las responsabilidades de éste, se cuenta el manejar a la perfección la cultura y perfil de la empresa, porque él es la voz de la compañía en las distintas redes sociales. Su trabajo no es solo escribir comentarios que fluyan, él debe ser el reflejo de la corporación para la cual trabaja, por ese motivo se necesita un profesional competente que sepa ser un aporte.

El nuevo cliente interactúa y se comunica con las empresas a través de las redes sociales, haciendo comentarios, preguntas y reportando problemas, retroalimentando a la empresa sobre sus productos y servicios. De ahí la importancia del trabajo del “community manager”, quien tiene entre sus actividades principales las siguientes:
\begin{itemize}
    \item Crear presencia en las redes sociales.
    \item Establecer y conservar fidelidad a la marca, productos o servicios.
    \item Investigar sobre la competencia.
    \item Crear estrategias de publicidad y comunicación.
    \item Segmentar al público objetivo de acuerdo a la propia plataforma de comunicación ya sea Facebook, Twitter, Linkedin, etc.
    \item Conseguir que el público de interés siga a la compañía en las diferentes redes sociales.
\end{itemize}

\section{Funciones del community manager}

Las funciones principales de un community manager son:
\begin{itemize}
    \item Diseñar la estrategia de redes sociales. El community manager debe diseñar una estrategia de redes sociales que involucre actividades como determinación de objetivos, definición de audiencias, definición de canales de redes sociales a participar, contenido a compartir por cada canal, etc.
    \item Definir metas y estrategias de crecimiento. El community manager debe definir las metas de crecimiento de cada canal y estrategias de crecimiento de las mismas.
    \item Gestión a la reputación en redes sociales. El community manager debe hacer gestión a la reputación en redes sociales de la empresa o marca que representa, con actividades como: monitorización de la reputación, creación del protocolo de crisis reputacional, construcción de respuestas en casos de crisis, etc.
    \item Diseñar el plan conversacional. El community manager debe crear un programa anual de comunicación en redes sociales, que responda a las necesidades de la empresa y de la audiencia.
    \item Crear contenido para redes sociales. El community manager debe tener la capacidad de crear contenido de valor agregado para cada audiencia y para cada red social.
    \item Generar conversación. El community manager debe generar conversación con su audiencia, provocando la participación en los diferentes canales de redes sociales.
    \item Conseguir relacionamiento. Con un tono cercano e información de valor agregado el community manager debe ganarse el corazón de su audiencia.
\end{itemize}

\section{Planificación de contenidos}

Las redes sociales han transformando la forma en la que las marcas se comunican con los usuarios, rompiendo la barrera que les separaba y poniendo a su disposición una plataforma ideal para establecer una conversación de tú a tú. Por ello, las marcas ya no ven a las redes sociales solo como un canal de comunicación, sino también como algo que les puede ayudar a vender. Por ello, se necesita un plan de contenidos es la guía o documento que ayuda a saber qué contenido se publican en los canales para conseguir llegar al público y los objetivos que se han marcado. 

El plan de contenidos es el documento base para poder diseñar, crear y programar los contenidos de la estrategia. Un documento que ayuda a poner en marcha la estrategia de contenidos del plan social media marketing y será imprescindible para optimizar los resultados. Primero de todo, para realizar el plan, se necesita analizar e invertir tiempo. Detrás de una estrategia de contenidos de redes sociales hay mucho trabajo y, muchas veces, los resultados no son inmediatos, esto es importante tenerlo claro. Es necesario hacer un exhaustivo análisis si se quiere realizar una estrategia de contenidos de calidad, con la que se consigan los resultados que se desean. A continuación, se explican todos los pasos que se debe de dar para poder empezar con un plan. 

\subsection{Auditoría}

Primeramente se tiene que realizar una auditoría para conocer la situación de la marca y de los competidores. Se debe recoger la situación actual del negocio o entidad en redes sociales y la de la competencia existente. Por lo que se realizará una auditoría interna y externa. 

\begin{itemize}
    \item \textbf{Análisis interno}. Se trata de comprobar la situación de la empresa o marca en redes sociales, respecto a la competencia. Este análisis ofrece una foto de la situación actual y de esta forma, se podrá elegir los contenidos adecuados y  optimizar así la presencia en redes sociales. Para realizar el análisis interno de las redes, se debe realizar un informe recogiendo todos los datos de la presencia en redes: En qué redes sociales se está, cuál es el posicionamiento, qué problemas impiden posicionar frente a la competencia, cuál es la frecuencia de publicación, cuáles son los contenidos con más engagement, cuándo se tiene más interacción, etc. Para realizar este análisis se puede utilizar las analíticas que ofrecen las propias redes sociales, o contar con herramientas externas que ayudan a obtener un análisis más exhaustivo.
    \item \textbf{Análisis externo}. Se tiene que identificar quiénes son los competidores y recoger también los datos de su presencia en redes sociales. En qué redes sociales está la competencia, cuál es su frecuencia de publicación, qué contenidos les está funcionando, etc.
\end{itemize}

\subsection{DAFO}

La auditoría realizada previamente ayudará a elaborar el análisis DAFO en social media. Antes de empezar a realizar un plan de contenidos para redes sociales conviene realizar este análisis. El análisis DAFO social media será fundamental para desarrollar un plan. A continuación se indican los puntos a analizar para elaborar un DAFO:

\begin{itemize}
    \item \textbf{Debilidades}. Analiza las razones detrás de los problemas que has identificado en la auditoría. Por ejemplo: una debilidad puede ser la falta de experiencia en social media del personal, o la falta de recursos o herramientas para elaborar contenidos para redes. 
    \item \textbf{Fortalezas}. A partir del análisis interno, es necesario analizar qué se hace mejor que la competencia: cuál es el elemento diferenciador. Por ejemplo: una fortaleza podría ser la alta preparación en social media del personal y contar con recursos para poner en marcha acciones en redes sociales.
    \item \textbf{Amenazas}. A partir del análisis externo, analizar por qué la competencia lo hace mejor que la marca: identificar los obstáculos que imposibilitan avanzar en el negocio. Por ejemplo: podría ser la alta preparación en social media de la competencia, frente a la baja preparación del personal de la marca o las estrategias social media de los competidores.
    \item \textbf{Oportunidades}. Se analiza si el mercado para el que trabaja la marca está en crecimiento. Se debe identificar también si satisface la necesidad del público o cliente. Y también, si existen factores que favorezcan la presencia en redes sociales. Por ejemplo: podría ser que ninguna empresa de la competencia haya aprovechado el potencial de una determinada red social con la que se podría llegar al público objetivo. Si la competencia no ha sido rápida, es una oportunidad para ser los primeros en poner en marcha una estrategia social media.
\end{itemize}

Por tanto, en este análisis DAFO será fundamental conocer muy bien la empresa y modelo de negocio, pero también a los competidores existentes. Y con una buena gestión se debe convertir las debilidades en oportunidades.

\subsection{Definir al buyer persona}

Una vez realizada la auditoría y el DAFO, se debe identificar al Buyer Persona. Es decir, al cliente ideal. Cuanto más se conozca al cliente o persona usuaria de los servicios o productos de la marca, más fácil será poder llegar a ella. 

Lo ideal será realizar una ficha imaginando a esa persona a la que se dirige el mensaje: cómo es físicamente, qué sueños o aspiraciones tiene, cuál es su nivel de estudios, su nivel económico, etc. En este sentido, es importante realizar un estereotipo del cliente ideal. Este análisis ayudará a la hora de elaborar contenidos. Se podrá analizar mejor qué tipo de contenidos son los que desea, con qué lenguaje llegar a esa persona, tono  e incluso se podrá saber en qué red social se debe publicar ese contenido, la mejor hora de publicación, etc.

\subsection{Elegir las redes sociales}

No se trata de estar en todas, sino de elegir las redes sociales más adecuadas para el negocio o marca. Elegir las redes sociales para empresa adecuadas será básico para llegar al público y conseguir los objetivos establecidos. La base será realizar el análisis previo, que ayudará a elegir la red social adecuada para conseguir llegar al público.

\subsection{Definir objetivos}

Los objetivos social media del plan también marcarán los contenidos que se deben diseñar. Tener muy claro los objetivos que se quieren conseguir con la estrategia de contenidos, será imprescindible. Los objetivos deben de ser SMART: específicos, medibles, alcanzables y relevantes. 

\begin{itemize}
    \item \textbf{Específicos}, ¿Qué?. Se debe especificar los objetivos de manera clara. Cuanto más específico sea, mejor.
    \item \textbf{Medibles}, ¿Cuánto?. A la hora de establecer el objetivo se debe tener en cuenta que hay que medir ese objetivo para poderlo evaluar. Si no lo se puede medir, no sirve.
    \item \textbf{Alcanzable}, ¿Cómo?. Se debe definir objetivos realistas. Por esta razón, conocer la situación actual y la de la competencia será fundamental. Este análisis ayudará a establecer objetivos social media realistas y alcanzables. 
    \item \textbf{Relevante}, ¿Con qué?. Aunque se defina un objetivo específico, medible y alcanzable, si no está enfocado a lo que el negocio necesita, tampoco sirve.
    \item \textbf{Tiempo determinado}, ¿Cuándo?. Y por supuesto, estos objetivo tendrán que tener una fecha para su consecución. Es necesario establecer un plazo, un tiempo determinado para los objetivo social media.
\end{itemize}

Ejemplo:
\begin{itemize}
    \item “Aumentar un 10\% el tráfico de la página web en 3 meses”
    \item “vender 30 plazas del curso X  en el plazo de 2 meses”.
\end{itemize}

Los objetivos que se identifiquen para las redes sociales, normalmente, están relacionados con: posicionamiento web, branding/visibilidad, leads, conversiones, etc.

\subsection{Estrategia de contenidos}

Después de dar todos los pasos visto hasta el momento, ahora llega el momento de empezar a definir la estrategia y las acciones que se van a llevar a cabo. Para que la estrategia de contenidos sea efectiva, será necesario que antes se analicen todos los aspectos que se han visto.

Se tiene que definir qué estrategia se va a poner en marcha (de lanzamiento, de visibilidad, de confianza, de posicionamiento, de promoción, de expansión…). Y además, qué acciones social media van ayudar a conseguir esos objetivos.

\subsection{Plan de difusión y promoción}

El plan de contenidos también debe ir acompañado de un plan de difusión y promoción. Es importante como se ha visto analizar, diseñar la estrategia de contenidos, crear, programar y publicar, pero no se debe olvidar de la promoción.

Para diseñar este plan podemos contar con medios pagados (Ads, banners, patrocinios…)  o medios propios (blog, landing page, newsletter, etc.). También se puede contar con recursos gratuitos que ayudarán a aumentar la visibilidad de esos contenidos. Por ejemplo, notas de prensa, foros, grupos especializados, agregadores de noticias, post invitados, etc.

Lo más recomendable es que si se dispone de presupuesto se puede diseñar un plan de difusión que combine diferentes formas de promocionar contenidos: de pago y propias. Si no se cuenta con un presupuesto, diseñar un plan de difusión con medios propios y gratuitos que ayuden a promocionar los contenidos, y aumentar de esta forma, la visibilidad de éstos.

\subsection{Medición}

Es muy  importante que la estrategia cuente con un apartado de medición. Si no se mide, no se va a saber si los contenidos que se han diseñado, están ayudando a conseguir los objetivos marcados. Por tanto, se debe tener en cuenta además de analizar y definir estrategias y tácticas, cómo se va a medir los contenidos. Es decir, qué métricas se van a utilizar y los KPIS con los que se va a medir la efectividad del plan de contenidos.

Esta medición será la base de la estrategia. Si se está invirtiendo tiempo y dinero en analizar, diseñar y crear contenidos que luego no generan conversiones, no servirá de mucho. Si no se mide no se va a poder mejorar. Por tanto, analizar, medir y si es necesario, realizar ajustes y optimizar el plan para conseguir contenidos que de verdad ayuden a conseguir los objetivos. 

Realizar un plan de contenidos para las redes sociales requiere tiempo, análisis y mucho trabajo. Pero sin duda, será la base para que la estrategia social media funcione. Empezar a gestionar las redes sociales sin realizar este análisis, es dar palos de ciegos, lo que se traduce en pérdida de tiempo y dinero. Si se quiere apostar por una presencia profesional que convierta, será necesario este análisis y diseño de contenidos. 

\section{Curación de contenidos}

La curación de contenidos, aunque tiene un nombre poco conocido, se refiere a un concepto que ha existido por mucho tiempo. En concreto, la curación de contenidos se entiende como la capacidad por parte de un sistema o del ser humano de encontrar, organizar, filtrar y dotar de valor, relevancia, significatividad, en definitiva, de utilidad el contenido de un tema específico que procede de diversas fuentes (medios digitales, herramientas de comunicación, redes sociales…). La curación de contenidos, además de una actividad relacionada con el marketing, también es vista como una forma de aliviar la \href{https://www.webempresa.com/blog/que-es-infoxicacion.html}{infoxicación digital} que se sufre frecuentemente en los medios digitales.

\subsection{Curador de contenidos}

El curador de contenidos, en inglés content curator, es la persona que de forma permanente busca, reúne, organiza y comparte el contenido de un tema específico, es decir, realiza la labor de curación de contenido. 

El “curador de contenidos” se convierte en la persona que separa el grano de la paja y ahorra tiempo de su audiencia. En las empresas pequeñas y medianas esta labor es realizada por el community manager, en las empresas grandes donde las actividades de éste se empiezan a distribuir en varias personas, se puede tener una persona dedicada al tema.

El curador de contenidos debe tener buena capacidad de análisis y gran capacidad de síntesis para lograr el criterio necesario para identificar el contenido relevante. En este sentido, el trabajo de curador de contenido no es crear no consiste en crear contenido sino en potencializar los contenidos de otros para que sirvan como fuente de innovación y conocimiento. Por ello, las características que debe tener un curador de contenido son:
\begin{itemize}
    \item \textbf{Conocer la audiencia}. Debe conocer su audiencia, la marca y la categoría de la cual es curador.
    \item \textbf{Mantenerse informado}. Mantiene informado de las tendencias, iniciativas e innovaciones de un nicho de mercado específico.
    \item \textbf{Monitorea tendencias}. Presta atención a los artículos, entrevistas, vídeos y otras fuentes de información para descubrir las tendencias existentes (trending topics).
    \item \textbf{Buen criterio}. Tiene la capacidad de distinguir el contenido valioso del que no lo es entre las montañas de información de Internet y las redes sociales.
    \item \textbf{Seguir eventos}. Vigila de cerca los eventos y actividades académicas de la industria.
    \item \textbf{Intuición}. Reconoce las noticias desde que se producen cuando llegarán a ser importantes gracias a su gran intuición.
    \item \textbf{Curioso}. Siente curiosidad por toda la información que tiene que ver con su nicho de mercado y disfruta absorbiendo toda la información.
    \item \textbf{Conoce su responsabilidad}. Compila conocimiento valioso y creíble, sabiendo que sus lectores confían en su criterio y disfrutan de la mezcla de contenidos de calidad que usted reúne.
\end{itemize}

\subsection{Formas de curación}

La curación de contenido se puede llevar a cabo de las siguientes formas:
\begin{itemize}
    \item \textbf{Agregación}. Reunir la información más relevante sobre un tema específico en una misma localización.
    \item \textbf{Destilación}. Realizar la curación en un formato más simple, donde sólo las ideas más importantes o relevantes son compartidas.
    \item \textbf{Elevación}. Identificar tendencias en pequeñas porciones de información compartidas online (como los tuits).
    \item \textbf{Mezcla}. Mezcla de contenidos curados para crear un nuevo punto de vista.
    \item \textbf{Cronología}. Reunir información histórica organizada sobre la base de tiempo para mostrar la evolución en la percepción de un tema en particular.
\end{itemize}

\subsection{Beneficios de la curación de contenidos}

Algunos beneficios de la curación de contenidos son:

\begin{itemize}
    \item \textbf{Optimiza el posicionamiento en motores de búsqueda (SEO)}. La curación de contenidos permite generar contenido para un sitio web o blog lo que mejora el posicionamiento natural en los buscadores.
    \item \textbf{Mejora la reputación}. La curación de contenidos favorece el fortalecimiento de la marca (branding) y permite conseguir relevancia en la audiencia.
    \item \textbf{Aumenta la productividad}. La curación de contenidos permite ahorrar tiempo y esfuerzo, logrando trabajar de una forma ágil en la información que es relevante para la audiencia de la empresa.
    \item \textbf{Facilita el relacionamiento}. Curar contenidos facilita el relacionamiento entre autores de los artículos originales y las marcas que lo viralizan y permite establecer alianzas que fortalezcan la estrategia de comunicación y marketing de la empresa.
\end{itemize}

\subsection{Pasos del proceso de curación de contenidos}

El proceso de curación de contenidos involucra las siguientes etapas:

\begin{enumerate}
    \item \textbf{Identificar necesidades}. El proceso de adquisición de la información debe responder a una estrategia corporativa formulada con anterioridad. Se deben identificar con precisión las necesidades de la audiencia a la que se va a entregar el contenido recopilado y curado.
    \item \textbf{Adquirir contenido}. El curador de contenidos debe estar en capacidad de identificar la información en Internet y las redes sociales que sea valiosa para la audiencia y que esté alineada con la estrategia corporativa.
    \item \textbf{Validar y almacenar}. El curador debe validar la pertinencia, usabilidad y la actualidad del contenido encontrado en Internet y las redes sociales para luego almacenarlo con sus respectivas fuentes en un sistema de repositorio confiable.
    \item \textbf{Compartir y distribuir}. Se deben seleccionar los canales para compartir y distribuir el contenido. Las redes sociales son muy eficaces para esta etapa del proceso, en especial servicios como: Facebook, Twitter, LinkedIn, blogs, sitio Web, entre otros.
    \item \textbf{Medir y mejorar}. Se debe estar atento a la reacción de la audiencia con cada pieza de contenido para identificar cuál es de más interés para ellos. Se debe afinar de forma permanente el tipo de contenido seleccionado de acuerdo a esta retroalimentación.
\end{enumerate}

\subsection{Herramientas de la curación de contenidos}

Algunas de las principales herramientas para realizar curación de contenidos son las siguientes:

\begin{itemize}
    \item \href{https://paper.li/}{Paper.li}. Es una herramienta que permite crear un “periódico” digital a la  medida de cada usuario.
    \item \href{https://about.flipboard.com/}{Flipboard}. Es una aplicación para crear un feed de temas que permite añadir diariamente las últimas publicaciones de su interés.
    \item \href{https://getpocket.com}{Pocket}. Es una herramienta que permite almacenar los artículos o información relevante que quieras leer más tarde.
    \item \href{https://www.scoop.it}{Scoop.it}. Es similar a \href{https://paper.li}{Paper.li} o \href{https://www.exprimiendolinkedin.com/linkedin-pulse/}{LinkedIn Pulse}, permite compartir contenidos, filtrarlos y almacenarlos para leerlos.
    \item \href{https://www.ready4social.com}{Ready4Social}. Es un gestor de redes sociales con curación de contenidos.
    \item \href{https://feedly.com}{Feedly}. Es uno de los grandes lectores de feeds y más sencillo de utilizar que existe actualmente en Internet.
\end{itemize}
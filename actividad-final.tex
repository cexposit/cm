\documentclass{article}

\usepackage[utf8]{inputenc}
\usepackage{hyperref}
\renewcommand{\baselinestretch}{1.5} 


\title{
    Community Manager: Gestión, Desarrollo y Mantenimiento de Comunidades Virtuales \\
    \large Actividad de Evaluación}
\date{}
\begin{document}

\maketitle

La actividad consiste en la \textbf{entrega individual de un informe en formato PDF} de la extensión que consideren, teniendo en cuenta que \textbf{de 5 a 10 páginas} es lo que consideramos razonable. En este informe se deben incluir los apartados que describimos a continuación y todos aquellos adicionales que consideren oportuno incluir.

Una vez realizado el informe, envíalo por correo a las siguientes cuentas:
\begin{itemize}
    \item cexposit@ull.edu.es
    \item aexposim@ull.edu.es
\end{itemize}

Una gran parte del éxito de la labor de un community manager radica en su formación y su experiencia. En este sentido, en esta actividad les pedimos que analicen el trabajo llevado a cabo por gestores de comunidades virtuales con anterioridad y extraigan conocimiento de éste. En concreto, poniéndonos en el papel de un Community Manager que va realizar labores de gestión de las comunidades virtuales de una marca, se pide realizar lo siguiente: 
\begin{itemize}
    \item Identificar una empresa con presencia en los medios sociales y describir su modelo de negocio.
    \item Identificar alguna campaña, acción, actividad continuada, etc. que haya llevado a cabo la empresa elegida en los medios sociales y en la que intervenga un community manager. En relación a este aspecto:
    \begin{itemize}
        \item Descripción de la empresa antes de la campaña.
        \item Descripción de la campaña.
        \item Objetivos de la campaña: ¿qué esperaba obtener la empresa?
        \item Público objetivo de la campaña: edad, género, ubicación geográfica, etc. 
        \item Inversión realizada y duración.
        \item Resultados de la campaña: ¿se alcanzaron los objetivos? ¿en qué medida? ¿fue un éxito/fracaso? 
    \end{itemize}
    \item Conclusiones personales. Este punto debe contener un conjunto de reflexiones generales a modo resumen del estado de la marca y de su labor comercial con respecto a sus campañas publicitarias.
\end{itemize}

\end{document}
